% Options for packages loaded elsewhere
\PassOptionsToPackage{unicode}{hyperref}
\PassOptionsToPackage{hyphens}{url}
%
\documentclass[
]{article}
\usepackage{amsmath,amssymb}
\usepackage{iftex}
\ifPDFTeX
  \usepackage[T1]{fontenc}
  \usepackage[utf8]{inputenc}
  \usepackage{textcomp} % provide euro and other symbols
\else % if luatex or xetex
  \usepackage{unicode-math} % this also loads fontspec
  \defaultfontfeatures{Scale=MatchLowercase}
  \defaultfontfeatures[\rmfamily]{Ligatures=TeX,Scale=1}
\fi
\usepackage{lmodern}
\ifPDFTeX\else
  % xetex/luatex font selection
\fi
% Use upquote if available, for straight quotes in verbatim environments
\IfFileExists{upquote.sty}{\usepackage{upquote}}{}
\IfFileExists{microtype.sty}{% use microtype if available
  \usepackage[]{microtype}
  \UseMicrotypeSet[protrusion]{basicmath} % disable protrusion for tt fonts
}{}
\makeatletter
\@ifundefined{KOMAClassName}{% if non-KOMA class
  \IfFileExists{parskip.sty}{%
    \usepackage{parskip}
  }{% else
    \setlength{\parindent}{0pt}
    \setlength{\parskip}{6pt plus 2pt minus 1pt}}
}{% if KOMA class
  \KOMAoptions{parskip=half}}
\makeatother
\usepackage{xcolor}
\usepackage[margin=1in]{geometry}
\usepackage{color}
\usepackage{fancyvrb}
\newcommand{\VerbBar}{|}
\newcommand{\VERB}{\Verb[commandchars=\\\{\}]}
\DefineVerbatimEnvironment{Highlighting}{Verbatim}{commandchars=\\\{\}}
% Add ',fontsize=\small' for more characters per line
\usepackage{framed}
\definecolor{shadecolor}{RGB}{248,248,248}
\newenvironment{Shaded}{\begin{snugshade}}{\end{snugshade}}
\newcommand{\AlertTok}[1]{\textcolor[rgb]{0.94,0.16,0.16}{#1}}
\newcommand{\AnnotationTok}[1]{\textcolor[rgb]{0.56,0.35,0.01}{\textbf{\textit{#1}}}}
\newcommand{\AttributeTok}[1]{\textcolor[rgb]{0.13,0.29,0.53}{#1}}
\newcommand{\BaseNTok}[1]{\textcolor[rgb]{0.00,0.00,0.81}{#1}}
\newcommand{\BuiltInTok}[1]{#1}
\newcommand{\CharTok}[1]{\textcolor[rgb]{0.31,0.60,0.02}{#1}}
\newcommand{\CommentTok}[1]{\textcolor[rgb]{0.56,0.35,0.01}{\textit{#1}}}
\newcommand{\CommentVarTok}[1]{\textcolor[rgb]{0.56,0.35,0.01}{\textbf{\textit{#1}}}}
\newcommand{\ConstantTok}[1]{\textcolor[rgb]{0.56,0.35,0.01}{#1}}
\newcommand{\ControlFlowTok}[1]{\textcolor[rgb]{0.13,0.29,0.53}{\textbf{#1}}}
\newcommand{\DataTypeTok}[1]{\textcolor[rgb]{0.13,0.29,0.53}{#1}}
\newcommand{\DecValTok}[1]{\textcolor[rgb]{0.00,0.00,0.81}{#1}}
\newcommand{\DocumentationTok}[1]{\textcolor[rgb]{0.56,0.35,0.01}{\textbf{\textit{#1}}}}
\newcommand{\ErrorTok}[1]{\textcolor[rgb]{0.64,0.00,0.00}{\textbf{#1}}}
\newcommand{\ExtensionTok}[1]{#1}
\newcommand{\FloatTok}[1]{\textcolor[rgb]{0.00,0.00,0.81}{#1}}
\newcommand{\FunctionTok}[1]{\textcolor[rgb]{0.13,0.29,0.53}{\textbf{#1}}}
\newcommand{\ImportTok}[1]{#1}
\newcommand{\InformationTok}[1]{\textcolor[rgb]{0.56,0.35,0.01}{\textbf{\textit{#1}}}}
\newcommand{\KeywordTok}[1]{\textcolor[rgb]{0.13,0.29,0.53}{\textbf{#1}}}
\newcommand{\NormalTok}[1]{#1}
\newcommand{\OperatorTok}[1]{\textcolor[rgb]{0.81,0.36,0.00}{\textbf{#1}}}
\newcommand{\OtherTok}[1]{\textcolor[rgb]{0.56,0.35,0.01}{#1}}
\newcommand{\PreprocessorTok}[1]{\textcolor[rgb]{0.56,0.35,0.01}{\textit{#1}}}
\newcommand{\RegionMarkerTok}[1]{#1}
\newcommand{\SpecialCharTok}[1]{\textcolor[rgb]{0.81,0.36,0.00}{\textbf{#1}}}
\newcommand{\SpecialStringTok}[1]{\textcolor[rgb]{0.31,0.60,0.02}{#1}}
\newcommand{\StringTok}[1]{\textcolor[rgb]{0.31,0.60,0.02}{#1}}
\newcommand{\VariableTok}[1]{\textcolor[rgb]{0.00,0.00,0.00}{#1}}
\newcommand{\VerbatimStringTok}[1]{\textcolor[rgb]{0.31,0.60,0.02}{#1}}
\newcommand{\WarningTok}[1]{\textcolor[rgb]{0.56,0.35,0.01}{\textbf{\textit{#1}}}}
\usepackage{graphicx}
\makeatletter
\def\maxwidth{\ifdim\Gin@nat@width>\linewidth\linewidth\else\Gin@nat@width\fi}
\def\maxheight{\ifdim\Gin@nat@height>\textheight\textheight\else\Gin@nat@height\fi}
\makeatother
% Scale images if necessary, so that they will not overflow the page
% margins by default, and it is still possible to overwrite the defaults
% using explicit options in \includegraphics[width, height, ...]{}
\setkeys{Gin}{width=\maxwidth,height=\maxheight,keepaspectratio}
% Set default figure placement to htbp
\makeatletter
\def\fps@figure{htbp}
\makeatother
\setlength{\emergencystretch}{3em} % prevent overfull lines
\providecommand{\tightlist}{%
  \setlength{\itemsep}{0pt}\setlength{\parskip}{0pt}}
\setcounter{secnumdepth}{-\maxdimen} % remove section numbering
\ifLuaTeX
  \usepackage{selnolig}  % disable illegal ligatures
\fi
\usepackage{bookmark}
\IfFileExists{xurl.sty}{\usepackage{xurl}}{} % add URL line breaks if available
\urlstyle{same}
\hypersetup{
  pdftitle={PEC1\_Datos\_Omicos},
  pdfauthor={Raul},
  hidelinks,
  pdfcreator={LaTeX via pandoc}}

\title{PEC1\_Datos\_Omicos}
\author{Raul}
\date{2024-11-06}

\begin{document}
\maketitle

{
\setcounter{tocdepth}{3}
\tableofcontents
}
\begin{Shaded}
\begin{Highlighting}[]
\FunctionTok{install.packages}\NormalTok{(}\StringTok{"readxl"}\NormalTok{)}
\ControlFlowTok{if}\NormalTok{ (}\SpecialCharTok{!}\FunctionTok{requireNamespace}\NormalTok{(}\StringTok{"BiocManager"}\NormalTok{, }\AttributeTok{quietly =} \ConstantTok{TRUE}\NormalTok{))}
    \FunctionTok{install.packages}\NormalTok{(}\StringTok{"BiocManager"}\NormalTok{)}
\NormalTok{BiocManager}\SpecialCharTok{::}\FunctionTok{install}\NormalTok{(}\StringTok{"SummarizedExperiment"}\NormalTok{)}
\end{Highlighting}
\end{Shaded}

\textless\textless\textless\textless\textless\textless\textless{} HEAD

\section{Abstract}\label{abstract}

Este estudio emplea un análisis metabolómico exhaustivo para identificar
biomarcadores potenciales y explorar las alteraciones metabólicas
específicas en el cáncer gástrico (GC) frente a tumores benignos (BN) y
controles sanos (HE). Mediante resonancia magnética nuclear (RMN), se
analizaron metabolitos específicos, siguiendo un control de calidad para
garantizar la robustez de los datos. Aquellos metabolitos que mostraron
un porcentaje de datos faltantes menor al 10\% y una desviación estándar
relativa (RSD) en los controles de calidad inferior al 20\% fueron
incluidos en el análisis, asegurando así la precisión y consistencia de
los resultados.

El análisis estadístico comenzó con una exploración de la variabilidad
de los metabolitos entre los diferentes grupos. Se aplicaron pruebas de
homogeneidad de varianza, como el test de Bartlett, que identificó
metabolitos con variabilidad significativamente distinta entre grupos
(e.g., 1-Methylnicotinamide (M4) con 𝑝\textless4.2×10−7
p\textless4.2×10−7), sugiriendo alteraciones específicas en el perfil
metabólico de los pacientes con GC. Estos resultados preliminares
orientaron el enfoque hacia metabolitos específicos que presentan un
potencial discriminativo relevante en el contexto clínico.

En la evaluación de diferencias de abundancia entre grupos, se calculó
el fold change de cada metabolito, indicando aumentos significativos en
ciertos metabolitos para el GC en comparación con los controles sanos y
benignos. Por ejemplo, el metabolito M4 mostró un incremento
significativo en cáncer gástrico respecto al control sano (HE) con un
p-valor ajustado de 0.028, lo cual subraya su potencial como marcador
diferencial.

Para comprender mejor las relaciones entre los metabolitos y su
capacidad para distinguir entre los grupos clínicos, se realizaron
análisis multivariados, específicamente Análisis de Componentes
Principales (PCA) y Análisis Discriminante de Mínimos Cuadrados
Parciales (PLS-DA). El PCA mostró una separación clara entre los grupos
HE, BN y GC, evidenciando que las muestras poseen patrones de expresión
metabólica diferenciados. En el modelo PLS-DA, que permite una
separación más precisa entre clases, se identificaron metabolitos como
M138, M134 y M118 como altamente discriminativos, lo que sugiere que
estos compuestos pueden estar asociados con alteraciones metabólicas
características del cáncer gástrico. Los valores de clasificación y
validación cruzada en PLS-DA indicaron una robustez razonable del
modelo, con un error de clasificación medio de 0.52 para el componente
principal 1 (GC vs.~HE) y de 0.41 en el segundo componente.

Los resultados obtenidos de este estudio resaltan la presencia de
perfiles metabólicos alterados en pacientes con cáncer gástrico,
particularmente en metabolitos como M4, 2-Furoylglycine (M7) y u233
(M138), que presentan patrones de abundancia distintivos. Estos
resultados aportan evidencia preliminar de posibles biomarcadores que
podrían contribuir a mejorar el diagnóstico y la comprensión de la
fisiopatología del cáncer gástrico. No obstante, se requieren estudios
adicionales para validar estos hallazgos en cohortes de mayor tamaño y
en otros contextos clínicos, lo cual permitiría consolidar el papel de
estos metabolitos como indicadores específicos de GC.

Este análisis metabolómico proporciona una base significativa para
futuros estudios que busquen integrar datos de múltiples ómicas y
enfoques bioinformáticos avanzados para el desarrollo de herramientas
diagnósticas y pronósticas en cáncer gástrico.

\begin{Shaded}
\begin{Highlighting}[]
\CommentTok{\# Cargar las librerías}
\FunctionTok{library}\NormalTok{(readxl)}
\FunctionTok{library}\NormalTok{(SummarizedExperiment)}
\end{Highlighting}
\end{Shaded}

\begin{verbatim}
## Cargando paquete requerido: MatrixGenerics
\end{verbatim}

\begin{verbatim}
## Cargando paquete requerido: matrixStats
\end{verbatim}

\begin{verbatim}
## 
## Adjuntando el paquete: 'MatrixGenerics'
\end{verbatim}

\begin{verbatim}
## The following objects are masked from 'package:matrixStats':
## 
##     colAlls, colAnyNAs, colAnys, colAvgsPerRowSet, colCollapse,
##     colCounts, colCummaxs, colCummins, colCumprods, colCumsums,
##     colDiffs, colIQRDiffs, colIQRs, colLogSumExps, colMadDiffs,
##     colMads, colMaxs, colMeans2, colMedians, colMins, colOrderStats,
##     colProds, colQuantiles, colRanges, colRanks, colSdDiffs, colSds,
##     colSums2, colTabulates, colVarDiffs, colVars, colWeightedMads,
##     colWeightedMeans, colWeightedMedians, colWeightedSds,
##     colWeightedVars, rowAlls, rowAnyNAs, rowAnys, rowAvgsPerColSet,
##     rowCollapse, rowCounts, rowCummaxs, rowCummins, rowCumprods,
##     rowCumsums, rowDiffs, rowIQRDiffs, rowIQRs, rowLogSumExps,
##     rowMadDiffs, rowMads, rowMaxs, rowMeans2, rowMedians, rowMins,
##     rowOrderStats, rowProds, rowQuantiles, rowRanges, rowRanks,
##     rowSdDiffs, rowSds, rowSums2, rowTabulates, rowVarDiffs, rowVars,
##     rowWeightedMads, rowWeightedMeans, rowWeightedMedians,
##     rowWeightedSds, rowWeightedVars
\end{verbatim}

\begin{verbatim}
## Cargando paquete requerido: GenomicRanges
\end{verbatim}

\begin{verbatim}
## Cargando paquete requerido: stats4
\end{verbatim}

\begin{verbatim}
## Cargando paquete requerido: BiocGenerics
\end{verbatim}

\begin{verbatim}
## 
## Adjuntando el paquete: 'BiocGenerics'
\end{verbatim}

\begin{verbatim}
## The following objects are masked from 'package:stats':
## 
##     IQR, mad, sd, var, xtabs
\end{verbatim}

\begin{verbatim}
## The following objects are masked from 'package:base':
## 
##     anyDuplicated, aperm, append, as.data.frame, basename, cbind,
##     colnames, dirname, do.call, duplicated, eval, evalq, Filter, Find,
##     get, grep, grepl, intersect, is.unsorted, lapply, Map, mapply,
##     match, mget, order, paste, pmax, pmax.int, pmin, pmin.int,
##     Position, rank, rbind, Reduce, rownames, sapply, setdiff, table,
##     tapply, union, unique, unsplit, which.max, which.min
\end{verbatim}

\begin{verbatim}
## Cargando paquete requerido: S4Vectors
\end{verbatim}

\begin{verbatim}
## 
## Adjuntando el paquete: 'S4Vectors'
\end{verbatim}

\begin{verbatim}
## The following object is masked from 'package:utils':
## 
##     findMatches
\end{verbatim}

\begin{verbatim}
## The following objects are masked from 'package:base':
## 
##     expand.grid, I, unname
\end{verbatim}

\begin{verbatim}
## Cargando paquete requerido: IRanges
\end{verbatim}

\begin{verbatim}
## 
## Adjuntando el paquete: 'IRanges'
\end{verbatim}

\begin{verbatim}
## The following object is masked from 'package:grDevices':
## 
##     windows
\end{verbatim}

\begin{verbatim}
## Cargando paquete requerido: GenomeInfoDb
\end{verbatim}

\begin{verbatim}
## Cargando paquete requerido: Biobase
\end{verbatim}

\begin{verbatim}
## Welcome to Bioconductor
## 
##     Vignettes contain introductory material; view with
##     'browseVignettes()'. To cite Bioconductor, see
##     'citation("Biobase")', and for packages 'citation("pkgname")'.
\end{verbatim}

\begin{verbatim}
## 
## Adjuntando el paquete: 'Biobase'
\end{verbatim}

\begin{verbatim}
## The following object is masked from 'package:MatrixGenerics':
## 
##     rowMedians
\end{verbatim}

\begin{verbatim}
## The following objects are masked from 'package:matrixStats':
## 
##     anyMissing, rowMedians
\end{verbatim}

\section{\texorpdfstring{\emph{Objetivos del
Estudio}}{Objetivos del Estudio}}\label{objetivos-del-estudio}

El objetivo principal de este trabajo es realizar un análisis
exploratorio de unos datos de metabolomica, descargados de un
repositorio github, utilizando el programa estadístico R y las librerías
para análisis de datos ómicos integradas en Bioconductor.

Como objetivos específicos podemos señalar los siguientes:

\begin{enumerate}
\def\labelenumi{\arabic{enumi}.}
\item
  Identificar un conjunto de datos (``dataset'') de interés en la tabla
  proporcionada y descargarlo para crear, un objeto de clase
  SummarizedExperiment con los datos de expresión y sus meta-datos.
\item
  Llevar a cabo un análisis exploratorio de los datos que proporcionar
  una visión general de las variables y de los individuos.
\end{enumerate}

\section{\texorpdfstring{\emph{Materiales y
Métodos}}{Materiales y Métodos}}\label{materiales-y-muxe9todos}

Es recomendable evaluar la calidad de los datos y eliminar cualquier
metabolito que esté mal medido antes de realizar análisis estadísticos o
aplicar modelos de aprendizaje automático (Broadhurst et al., 2018). En
el caso del conjunto de datos de Cáncer Gástrico por RMN utilizado en
este ejemplo, ya hemos calculado algunas estadísticas básicas para cada
metabolito, las cuales se encuentran registradas en la tabla Peak. En
este cuaderno, solo se mantendrán los metabolitos que cumplan con los
siguientes requisitos:

Un QC-RSD inferior al 20\% Menos del 10\% de los valores están ausentes
Una vez que los datos sean depurados, se indicará la cantidad de picos
que quedan.

\section{\texorpdfstring{\emph{Resultados}}{Resultados}}\label{resultados}

\subsection{Estructura de los datos y del
estudio}\label{estructura-de-los-datos-y-del-estudio}

\begin{Shaded}
\begin{Highlighting}[]
\CommentTok{\# Leer las hojas del archivo Excel}

\NormalTok{file\_path }\OtherTok{\textless{}{-}} \StringTok{"GastricCancer\_NMR.xlsx"}
\NormalTok{data\_df }\OtherTok{\textless{}{-}} \FunctionTok{read\_excel}\NormalTok{(file\_path, }\AttributeTok{sheet =} \StringTok{"Data"}\NormalTok{)}
\NormalTok{metabolites\_df }\OtherTok{\textless{}{-}} \FunctionTok{read\_excel}\NormalTok{(file\_path, }\AttributeTok{sheet =} \StringTok{"Peak"}\NormalTok{)}
\end{Highlighting}
\end{Shaded}

\begin{Shaded}
\begin{Highlighting}[]
\CommentTok{\# Filtrar las muestras para eliminar las que tienen SampleType = "QC"}
\NormalTok{data\_df }\OtherTok{\textless{}{-}}\NormalTok{ data\_df[data\_df}\SpecialCharTok{$}\NormalTok{SampleType }\SpecialCharTok{!=} \StringTok{"QC"}\NormalTok{, ]}
\end{Highlighting}
\end{Shaded}

\begin{Shaded}
\begin{Highlighting}[]
\CommentTok{\# Extraer la matriz de datos de las columnas que empiezan con \textquotesingle{}M\textquotesingle{}}
\NormalTok{metabolite\_columns }\OtherTok{\textless{}{-}} \FunctionTok{grep}\NormalTok{(}\StringTok{"\^{}M"}\NormalTok{, }\FunctionTok{names}\NormalTok{(data\_df), }\AttributeTok{value =} \ConstantTok{TRUE}\NormalTok{)}
\NormalTok{data\_matrix }\OtherTok{\textless{}{-}} \FunctionTok{as.matrix}\NormalTok{(data\_df[, metabolite\_columns])}
\end{Highlighting}
\end{Shaded}

\begin{Shaded}
\begin{Highlighting}[]
\CommentTok{\# Transponer la matriz para que las muestras estén en las columnas}
\NormalTok{data\_matrix }\OtherTok{\textless{}{-}} \FunctionTok{t}\NormalTok{(data\_matrix)}
\end{Highlighting}
\end{Shaded}

\begin{Shaded}
\begin{Highlighting}[]
\CommentTok{\# Crear los metadatos de las muestras (SampleID, SampleType, Class) después del filtrado}
\NormalTok{colData }\OtherTok{\textless{}{-}} \FunctionTok{DataFrame}\NormalTok{(}
    \AttributeTok{SampleID =}\NormalTok{ data\_df}\SpecialCharTok{$}\NormalTok{SampleID,}
    \AttributeTok{SampleType =}\NormalTok{ data\_df}\SpecialCharTok{$}\NormalTok{SampleType,}
    \AttributeTok{Class =}\NormalTok{ data\_df}\SpecialCharTok{$}\NormalTok{Class}
\NormalTok{)}
\end{Highlighting}
\end{Shaded}

\begin{Shaded}
\begin{Highlighting}[]
\CommentTok{\# Crear los metadatos de los metabolitos (Name, Label)}
\NormalTok{rowData }\OtherTok{\textless{}{-}} \FunctionTok{DataFrame}\NormalTok{(}
    \AttributeTok{Name =}\NormalTok{ metabolites\_df}\SpecialCharTok{$}\NormalTok{Name,}
    \AttributeTok{Label =}\NormalTok{ metabolites\_df}\SpecialCharTok{$}\NormalTok{Label,}
    \AttributeTok{Perc\_missing =}\NormalTok{ metabolites\_df}\SpecialCharTok{$}\NormalTok{Perc\_missing,}
    \AttributeTok{QC\_RSD =}\NormalTok{ metabolites\_df}\SpecialCharTok{$}\NormalTok{QC\_RSD}
\NormalTok{)}
\end{Highlighting}
\end{Shaded}

\begin{Shaded}
\begin{Highlighting}[]
\CommentTok{\# Crear el objeto SummarizedExperiment}
\NormalTok{se }\OtherTok{\textless{}{-}} \FunctionTok{SummarizedExperiment}\NormalTok{(}
    \AttributeTok{assays =} \FunctionTok{list}\NormalTok{(}\AttributeTok{counts =}\NormalTok{ data\_matrix),}
    \AttributeTok{rowData =}\NormalTok{ rowData,}
    \AttributeTok{colData =}\NormalTok{ colData,}
    \AttributeTok{metadata =} \FunctionTok{list}\NormalTok{(}
        \AttributeTok{description =} \StringTok{"Columns M1 ... M149 describe metabolite concentrations. Column SampleType indicates whether the sample was a pooled QC or a study sample. Column Class indicates the clinical outcome observed for that individual: GC = Gastric Cancer, BN = Benign Tumor, HE = Healthy Control."}
\NormalTok{    )}
\NormalTok{)}
\end{Highlighting}
\end{Shaded}

\begin{Shaded}
\begin{Highlighting}[]
\CommentTok{\# Filtrar los metabolitos según los criterios de calidad}
\NormalTok{se\_filtered }\OtherTok{\textless{}{-}}\NormalTok{ se[}
  \FunctionTok{rowData}\NormalTok{(se)}\SpecialCharTok{$}\NormalTok{Perc\_missing }\SpecialCharTok{\textless{}} \DecValTok{10} \SpecialCharTok{\&} \FunctionTok{rowData}\NormalTok{(se)}\SpecialCharTok{$}\NormalTok{QC\_RSD }\SpecialCharTok{\textless{}} \DecValTok{20}\NormalTok{, }
\NormalTok{  ]}
\end{Highlighting}
\end{Shaded}

\begin{Shaded}
\begin{Highlighting}[]
\CommentTok{\# Mostrar el objeto}
\FunctionTok{print}\NormalTok{(se\_filtered)}
\end{Highlighting}
\end{Shaded}

\begin{verbatim}
## class: SummarizedExperiment 
## dim: 52 123 
## metadata(1): description
## assays(1): counts
## rownames(52): M4 M5 ... M148 M149
## rowData names(4): Name Label Perc_missing QC_RSD
## colnames: NULL
## colData names(3): SampleID SampleType Class
\end{verbatim}

\subsection{1. Análisis Descriptivo
Inicial}\label{anuxe1lisis-descriptivo-inicial}

\begin{Shaded}
\begin{Highlighting}[]
\CommentTok{\# Cargar las librerías necesarias}
\FunctionTok{library}\NormalTok{(SummarizedExperiment)}
\FunctionTok{library}\NormalTok{(ggplot2)}
\end{Highlighting}
\end{Shaded}

\subsection{2. Visualización de Datos}\label{visualizaciuxf3n-de-datos}

\subsection{3. Análisis de Variabilidad y Control de
Calidad}\label{anuxe1lisis-de-variabilidad-y-control-de-calidad}

\begin{enumerate}
\def\labelenumi{\arabic{enumi}.}
\tightlist
\item
  Análisis de Varianza (ANOVA) o Test de Bartlett: Comparar la
  variabilidad de los metabolitos entre grupos (p.~ej., control sano,
  tumor benigno, cáncer gástrico) para evaluar diferencias
  significativas en la dispersión.
\end{enumerate}

\begin{Shaded}
\begin{Highlighting}[]
\CommentTok{\# Extraer la matriz de datos y la información de los grupos}
\NormalTok{data\_matrix }\OtherTok{\textless{}{-}} \FunctionTok{assay}\NormalTok{(se\_filtered, }\StringTok{"counts"}\NormalTok{)}
\NormalTok{class\_labels }\OtherTok{\textless{}{-}} \FunctionTok{colData}\NormalTok{(se\_filtered)}\SpecialCharTok{$}\NormalTok{Class}
\end{Highlighting}
\end{Shaded}

\begin{Shaded}
\begin{Highlighting}[]
\CommentTok{\# Aplicar el Test de Bartlett a cada metabolito}
\NormalTok{bartlett\_results }\OtherTok{\textless{}{-}} \FunctionTok{apply}\NormalTok{(data\_matrix, }\DecValTok{1}\NormalTok{, }\ControlFlowTok{function}\NormalTok{(x) \{}
  \FunctionTok{bartlett.test}\NormalTok{(x }\SpecialCharTok{\textasciitilde{}}\NormalTok{ class\_labels)}\SpecialCharTok{$}\NormalTok{p.value}
\NormalTok{\})}

\CommentTok{\# Mostrar los resultados de los primeros 10 metabolitos}
\NormalTok{bartlett\_results\_df }\OtherTok{\textless{}{-}} \FunctionTok{data.frame}\NormalTok{(}
  \AttributeTok{Metabolite =} \FunctionTok{rownames}\NormalTok{(data\_matrix),}
  \AttributeTok{P\_Value =}\NormalTok{ bartlett\_results}
\NormalTok{)}
\FunctionTok{print}\NormalTok{(bartlett\_results\_df)}
\end{Highlighting}
\end{Shaded}

\begin{verbatim}
##      Metabolite      P_Value
## M4           M4 4.224673e-07
## M5           M5 2.427859e-07
## M7           M7 4.324870e-05
## M8           M8 3.259681e-05
## M11         M11 8.126121e-09
## M14         M14 1.526602e-02
## M15         M15 2.570469e-01
## M25         M25 2.987421e-06
## M26         M26 6.198063e-20
## M31         M31 1.427729e-07
## M32         M32 1.907688e-07
## M33         M33 3.845738e-03
## M36         M36 2.108317e-01
## M37         M37 9.197993e-08
## M45         M45 6.224370e-03
## M48         M48 8.383434e-03
## M50         M50 5.699905e-47
## M51         M51 1.054609e-14
## M65         M65 1.640653e-02
## M66         M66 3.558642e-12
## M68         M68 1.967239e-06
## M71         M71 6.204509e-07
## M73         M73 2.129246e-01
## M74         M74 3.653292e-02
## M75         M75 1.490430e-04
## M88         M88 1.859847e-01
## M89         M89 4.384363e-12
## M90         M90 1.782541e-03
## M91         M91 4.335689e-03
## M93         M93 2.492443e-05
## M101       M101 2.014675e-02
## M104       M104 5.701262e-02
## M105       M105 1.201645e-04
## M106       M106 1.100085e-03
## M107       M107 1.610330e-01
## M110       M110 1.067979e-01
## M115       M115 7.156591e-11
## M116       M116 5.761903e-07
## M118       M118 1.546278e-12
## M119       M119 1.574691e-01
## M120       M120 4.857938e-06
## M122       M122 1.166833e-03
## M126       M126 1.025456e-21
## M129       M129 1.701761e-06
## M130       M130 2.870331e-11
## M134       M134 4.764165e-12
## M137       M137 2.922963e-02
## M138       M138 3.216760e-08
## M142       M142 2.185937e-09
## M144       M144 9.158809e-60
## M148       M148 9.873685e-02
## M149       M149 3.158070e-01
\end{verbatim}

Interpretación: Si un p-valor es menor que 0.05, sugiere que hay una
diferencia significativa en la variabilidad de ese metabolito entre los
grupos.

\begin{Shaded}
\begin{Highlighting}[]
\CommentTok{\# Aplicar ANOVA a cada metabolito}
\NormalTok{anova\_results }\OtherTok{\textless{}{-}} \FunctionTok{apply}\NormalTok{(data\_matrix, }\DecValTok{1}\NormalTok{, }\ControlFlowTok{function}\NormalTok{(x) \{}
  \FunctionTok{summary}\NormalTok{(}\FunctionTok{aov}\NormalTok{(x }\SpecialCharTok{\textasciitilde{}}\NormalTok{ class\_labels))[[}\DecValTok{1}\NormalTok{]][[}\StringTok{"Pr(\textgreater{}F)"}\NormalTok{]][}\DecValTok{1}\NormalTok{]}
\NormalTok{\})}

\CommentTok{\# Mostrar los resultados de los primeros 10 metabolitos}
\NormalTok{anova\_results\_df }\OtherTok{\textless{}{-}} \FunctionTok{data.frame}\NormalTok{(}
  \AttributeTok{Metabolite =} \FunctionTok{rownames}\NormalTok{(data\_matrix),}
  \AttributeTok{P\_Value =}\NormalTok{ anova\_results}
\NormalTok{)}
\FunctionTok{print}\NormalTok{(anova\_results\_df)}
\end{Highlighting}
\end{Shaded}

\begin{verbatim}
##      Metabolite      P_Value
## M4           M4 2.926107e-03
## M5           M5 3.175123e-01
## M7           M7 5.760973e-03
## M8           M8 1.222709e-01
## M11         M11 6.220643e-01
## M14         M14 2.888005e-01
## M15         M15 8.900031e-01
## M25         M25 1.103668e-01
## M26         M26 1.260561e-01
## M31         M31 6.153650e-01
## M32         M32 7.122914e-03
## M33         M33 4.229971e-01
## M36         M36 6.373565e-01
## M37         M37 5.086230e-01
## M45         M45 2.232977e-03
## M48         M48 1.482196e-01
## M50         M50 4.641730e-01
## M51         M51 1.572472e-01
## M65         M65 1.868775e-01
## M66         M66 1.375558e-01
## M68         M68 3.670154e-01
## M71         M71 6.506552e-01
## M73         M73 1.249217e-01
## M74         M74 7.358403e-01
## M75         M75 2.083039e-01
## M88         M88 5.486211e-01
## M89         M89 1.570093e-03
## M90         M90 2.750177e-01
## M91         M91 1.016169e-01
## M93         M93 2.709336e-01
## M101       M101 7.280050e-01
## M104       M104 2.620641e-01
## M105       M105 4.351292e-01
## M106       M106 4.583989e-01
## M107       M107 9.029112e-01
## M110       M110 4.764870e-01
## M115       M115 1.965663e-01
## M116       M116 3.122889e-01
## M118       M118 6.206694e-04
## M119       M119 6.760862e-01
## M120       M120 6.842801e-01
## M122       M122 7.415736e-01
## M126       M126 1.020387e-02
## M129       M129 4.528492e-01
## M130       M130 3.862113e-01
## M134       M134 5.641083e-04
## M137       M137 8.660670e-01
## M138       M138 4.078888e-05
## M142       M142 1.110574e-01
## M144       M144 3.438253e-01
## M148       M148 3.601856e-01
## M149       M149 2.819257e-01
\end{verbatim}

Un p-valor bajo (\textless{} 0.05) indica diferencias significativas en
las medias de las concentraciones entre los grupos

\begin{enumerate}
\def\labelenumi{\arabic{enumi}.}
\setcounter{enumi}{1}
\tightlist
\item
  PCA (Análisis de Componentes Principales):
\end{enumerate}

\begin{itemize}
\tightlist
\item
  Realizar un PCA para visualizar la separación global entre las clases
  (GC, BN, HE) y evaluar la presencia de agrupamientos claros o
  patrones.
\item
  Este paso también ayuda a identificar la necesidad de normalización
  adicional si hay un sesgo por efecto de lote o variabilidad técnica.
\end{itemize}

El PCA se usa para reducir la dimensionalidad de los datos y visualizar
cómo se agrupan las muestras según los metabolitos.

\begin{Shaded}
\begin{Highlighting}[]
\CommentTok{\# Función para imputar valores faltantes con la mediana de cada fila (metabolito)}
\NormalTok{data\_matrix }\OtherTok{\textless{}{-}} \FunctionTok{assay}\NormalTok{(se\_filtered, }\StringTok{"counts"}\NormalTok{)}
\NormalTok{data\_matrix }\OtherTok{\textless{}{-}} \FunctionTok{apply}\NormalTok{(data\_matrix, }\DecValTok{1}\NormalTok{, }\ControlFlowTok{function}\NormalTok{(x) \{}
\NormalTok{  x[}\FunctionTok{is.na}\NormalTok{(x) }\SpecialCharTok{|} \FunctionTok{is.infinite}\NormalTok{(x)] }\OtherTok{\textless{}{-}} \FunctionTok{median}\NormalTok{(x, }\AttributeTok{na.rm =} \ConstantTok{TRUE}\NormalTok{)}
  \FunctionTok{return}\NormalTok{(x)}
\NormalTok{\})}
\NormalTok{data\_matrix }\OtherTok{\textless{}{-}} \FunctionTok{t}\NormalTok{(data\_matrix) }
\end{Highlighting}
\end{Shaded}

\begin{Shaded}
\begin{Highlighting}[]
\CommentTok{\# Cargar la librería para el PCA}
\FunctionTok{library}\NormalTok{(ggplot2)}

\CommentTok{\# Transponer la matriz de datos para que las muestras sean filas y los metabolitos columnas}
\NormalTok{pca\_data }\OtherTok{\textless{}{-}} \FunctionTok{t}\NormalTok{(data\_matrix)}

\CommentTok{\# Realizar el PCA}
\NormalTok{pca\_result }\OtherTok{\textless{}{-}} \FunctionTok{prcomp}\NormalTok{(pca\_data, }\AttributeTok{scale. =} \ConstantTok{TRUE}\NormalTok{)  }\CommentTok{\# scale. = TRUE para normalizar los datos}

\CommentTok{\# Crear un data frame con los resultados del PCA}
\NormalTok{pca\_df }\OtherTok{\textless{}{-}} \FunctionTok{data.frame}\NormalTok{(}
  \AttributeTok{PC1 =}\NormalTok{ pca\_result}\SpecialCharTok{$}\NormalTok{x[, }\DecValTok{1}\NormalTok{],}
  \AttributeTok{PC2 =}\NormalTok{ pca\_result}\SpecialCharTok{$}\NormalTok{x[, }\DecValTok{2}\NormalTok{],}
  \AttributeTok{Class =}\NormalTok{ class\_labels}
\NormalTok{)}

\CommentTok{\# Gráfico del PCA}
\FunctionTok{ggplot}\NormalTok{(pca\_df, }\FunctionTok{aes}\NormalTok{(}\AttributeTok{x =}\NormalTok{ PC1, }\AttributeTok{y =}\NormalTok{ PC2, }\AttributeTok{color =}\NormalTok{ Class)) }\SpecialCharTok{+}
  \FunctionTok{geom\_point}\NormalTok{(}\AttributeTok{size =} \DecValTok{3}\NormalTok{) }\SpecialCharTok{+}
  \FunctionTok{theme\_minimal}\NormalTok{() }\SpecialCharTok{+}
  \FunctionTok{labs}\NormalTok{(}
    \AttributeTok{title =} \StringTok{"Análisis de Componentes Principales (PCA)"}\NormalTok{,}
    \AttributeTok{x =} \StringTok{"Componente Principal 1"}\NormalTok{,}
    \AttributeTok{y =} \StringTok{"Componente Principal 2"}
\NormalTok{  ) }\SpecialCharTok{+}
  \FunctionTok{scale\_color\_manual}\NormalTok{(}\AttributeTok{values =} \FunctionTok{c}\NormalTok{(}\StringTok{"red"}\NormalTok{, }\StringTok{"blue"}\NormalTok{, }\StringTok{"green"}\NormalTok{))}
\end{Highlighting}
\end{Shaded}

\includegraphics{PEC1_Datos_Omicos_files/figure-latex/unnamed-chunk-18-1.pdf}
\#\# Análisis Univariado Compararemos las concentraciones de cada
metabolito entre las clases utilizando la prueba de t de Student o la
prueba de Mann-Whitney U. Luego, aplicaremos una corrección por
múltiples comparaciones

\begin{Shaded}
\begin{Highlighting}[]
\FunctionTok{install.packages}\NormalTok{(}\StringTok{"dplyr"}\NormalTok{)}
\end{Highlighting}
\end{Shaded}

\begin{Shaded}
\begin{Highlighting}[]
\FunctionTok{library}\NormalTok{(dplyr)}
\end{Highlighting}
\end{Shaded}

\begin{verbatim}
## 
## Adjuntando el paquete: 'dplyr'
\end{verbatim}

\begin{verbatim}
## The following object is masked from 'package:Biobase':
## 
##     combine
\end{verbatim}

\begin{verbatim}
## The following objects are masked from 'package:GenomicRanges':
## 
##     intersect, setdiff, union
\end{verbatim}

\begin{verbatim}
## The following object is masked from 'package:GenomeInfoDb':
## 
##     intersect
\end{verbatim}

\begin{verbatim}
## The following objects are masked from 'package:IRanges':
## 
##     collapse, desc, intersect, setdiff, slice, union
\end{verbatim}

\begin{verbatim}
## The following objects are masked from 'package:S4Vectors':
## 
##     first, intersect, rename, setdiff, setequal, union
\end{verbatim}

\begin{verbatim}
## The following objects are masked from 'package:BiocGenerics':
## 
##     combine, intersect, setdiff, union
\end{verbatim}

\begin{verbatim}
## The following object is masked from 'package:matrixStats':
## 
##     count
\end{verbatim}

\begin{verbatim}
## The following objects are masked from 'package:stats':
## 
##     filter, lag
\end{verbatim}

\begin{verbatim}
## The following objects are masked from 'package:base':
## 
##     intersect, setdiff, setequal, union
\end{verbatim}

\begin{Shaded}
\begin{Highlighting}[]
\CommentTok{\# Extraer los datos}
\NormalTok{data\_matrix }\OtherTok{\textless{}{-}} \FunctionTok{assay}\NormalTok{(se\_filtered, }\StringTok{"counts"}\NormalTok{)}
\NormalTok{class\_labels }\OtherTok{\textless{}{-}} \FunctionTok{colData}\NormalTok{(se\_filtered)}\SpecialCharTok{$}\NormalTok{Class}

\CommentTok{\# Inicializar un data frame para almacenar los resultados}
\NormalTok{results }\OtherTok{\textless{}{-}} \FunctionTok{data.frame}\NormalTok{(}\AttributeTok{Metabolite =} \FunctionTok{rownames}\NormalTok{(data\_matrix), }\AttributeTok{p\_GC\_HE =} \ConstantTok{NA}\NormalTok{, }\AttributeTok{p\_GC\_BN =} \ConstantTok{NA}\NormalTok{)}

\CommentTok{\# Realizar las comparaciones entre grupos}
\ControlFlowTok{for}\NormalTok{ (i }\ControlFlowTok{in} \DecValTok{1}\SpecialCharTok{:}\FunctionTok{nrow}\NormalTok{(data\_matrix)) \{}
  \CommentTok{\# Extraer concentraciones del metabolito actual}
\NormalTok{  metabolite\_data }\OtherTok{\textless{}{-}}\NormalTok{ data\_matrix[i, ]}
  
  \CommentTok{\# Comparación entre GC y HE}
\NormalTok{  results}\SpecialCharTok{$}\NormalTok{p\_GC\_HE[i] }\OtherTok{\textless{}{-}} \FunctionTok{wilcox.test}\NormalTok{(metabolite\_data[class\_labels }\SpecialCharTok{==} \StringTok{"GC"}\NormalTok{], }
\NormalTok{                                     metabolite\_data[class\_labels }\SpecialCharTok{==} \StringTok{"HE"}\NormalTok{])}\SpecialCharTok{$}\NormalTok{p.value}
  
  \CommentTok{\# Comparación entre GC y BN}
\NormalTok{  results}\SpecialCharTok{$}\NormalTok{p\_GC\_BN[i] }\OtherTok{\textless{}{-}} \FunctionTok{wilcox.test}\NormalTok{(metabolite\_data[class\_labels }\SpecialCharTok{==} \StringTok{"GC"}\NormalTok{], }
\NormalTok{                                     metabolite\_data[class\_labels }\SpecialCharTok{==} \StringTok{"BN"}\NormalTok{])}\SpecialCharTok{$}\NormalTok{p.value}
\NormalTok{\}}

\CommentTok{\# Aplicar corrección por múltiples comparaciones (FDR de Benjamini{-}Hochberg)}
\NormalTok{results }\OtherTok{\textless{}{-}}\NormalTok{ results }\SpecialCharTok{\%\textgreater{}\%}
  \FunctionTok{mutate}\NormalTok{(}
    \AttributeTok{p\_adj\_GC\_HE =} \FunctionTok{p.adjust}\NormalTok{(p\_GC\_HE, }\AttributeTok{method =} \StringTok{"BH"}\NormalTok{),}
    \AttributeTok{p\_adj\_GC\_BN =} \FunctionTok{p.adjust}\NormalTok{(p\_GC\_BN, }\AttributeTok{method =} \StringTok{"BH"}\NormalTok{)}
\NormalTok{  )}

\CommentTok{\# Mostrar los primeros resultados}
\FunctionTok{print}\NormalTok{(results)}
\end{Highlighting}
\end{Shaded}

\begin{verbatim}
##    Metabolite      p_GC_HE      p_GC_BN  p_adj_GC_HE p_adj_GC_BN
## 1          M4 4.229112e-03 0.0008962928 2.848735e-02  0.04660722
## 2          M5 3.026708e-01 0.7946699775 5.952753e-01  0.96708271
## 3          M7 4.901039e-03 0.0239424072 2.848735e-02  0.33621232
## 4          M8 4.900314e-02 0.0258624863 1.592602e-01  0.33621232
## 5         M11 7.325709e-01 0.7109812185 9.037709e-01  0.96096604
## 6         M14 1.380516e-01 0.0688759505 3.778253e-01  0.51164992
## 7         M15 6.717219e-01 0.2254644232 9.037709e-01  0.68965588
## 8         M25 7.629027e-02 0.6258443649 2.333585e-01  0.95717373
## 9         M26 2.331754e-01 0.9283069739 5.271791e-01  0.99638411
## 10        M31 9.781868e-01 0.7207245276 9.973670e-01  0.96096604
## 11        M32 4.930502e-03 0.1803395156 2.848735e-02  0.68912042
## 12        M33 5.464179e-01 0.9674657614 8.610221e-01  0.99638411
## 13        M36 2.614856e-01 0.6237016753 5.665522e-01  0.95717373
## 14        M37 9.636541e-01 0.5294535766 9.973670e-01  0.88811568
## 15        M45 1.903161e-03 0.4219280282 1.979287e-02  0.84385606
## 16        M48 1.768740e-01 0.1367737949 4.211816e-01  0.68912042
## 17        M50 6.917799e-01 0.7841530415 9.037709e-01  0.96708271
## 18        M51 6.952387e-01 0.1615040572 9.037709e-01  0.68912042
## 19        M65 6.985140e-01 0.1848310963 9.037709e-01  0.68912042
## 20        M66 5.235058e-01 0.9963841054 8.506969e-01  0.99638411
## 21        M68 8.565865e-01 0.7997030069 9.499488e-01  0.96708271
## 22        M71 4.799977e-01 0.1956116797 8.291136e-01  0.68912042
## 23        M73 9.178830e-02 0.9963841054 2.651662e-01  0.99638411
## 24        M74 7.647292e-01 0.5140603693 9.037709e-01  0.88811568
## 25        M75 4.942792e-01 0.3156787680 8.291136e-01  0.83482747
## 26        M88 1.000000e+00 0.2691903574 1.000000e+00  0.77766103
## 27        M89 1.837302e-05 0.4707569848 4.776986e-04  0.88092217
## 28        M90 8.589462e-01 0.1987847368 9.499488e-01  0.68912042
## 29        M91 4.295074e-02 0.5689498540 1.488959e-01  0.92454351
## 30        M93 9.491329e-01 0.2237300316 9.973670e-01  0.68965588
## 31       M101 1.585088e-01 0.4743427080 4.121230e-01  0.88092217
## 32       M104 3.204627e-01 0.5029545835 5.952753e-01  0.88811568
## 33       M105 6.025665e-01 0.9593073464 9.037709e-01  0.99638411
## 34       M106 7.359643e-01 0.3853049856 9.037709e-01  0.83482747
## 35       M107 6.535623e-01 0.8525873389 9.037709e-01  0.99638411
## 36       M110 2.271063e-02 0.9322469632 9.899537e-02  0.99638411
## 37       M115 1.781922e-01 0.7155922935 4.211816e-01  0.96096604
## 38       M116 8.346575e-01 0.4169485942 9.499488e-01  0.84385606
## 39       M118 9.996093e-04 0.9703994559 1.299492e-02  0.99638411
## 40       M119 3.568471e-01 0.7086582092 6.398637e-01  0.96096604
## 41       M120 9.892548e-03 0.6745610385 5.144125e-02  0.96096604
## 42       M122 7.566587e-01 0.3851279539 9.037709e-01  0.83482747
## 43       M126 3.946278e-02 0.0231141697 1.465761e-01  0.33621232
## 44       M129 6.733173e-01 0.8739662308 9.037709e-01  0.99638411
## 45       M130 2.913906e-02 0.0652119695 1.165562e-01  0.51164992
## 46       M134 1.406853e-04 0.0508401981 2.438545e-03  0.51164992
## 47       M137 3.205328e-01 0.3586951097 5.952753e-01  0.83482747
## 48       M138 8.335364e-09 0.0793265594 4.334389e-07  0.51562264
## 49       M142 2.901147e-03 0.3277100198 2.514328e-02  0.83482747
## 50       M144 3.051509e-01 0.7497147605 5.952753e-01  0.96708271
## 51       M148 2.284509e-02 0.3386046293 9.899537e-02  0.83482747
## 52       M149 8.768758e-01 0.1018612947 9.499488e-01  0.58853192
\end{verbatim}

Los valores ajustados (p\_adj\_GC\_HE y p\_adj\_GC\_BN) reflejan la
significancia después de la corrección por múltiples comparaciones. Un
valor ajustado bajo (\textless{} 0.05) sugiere una diferencia
significativa entre los grupos para ese metabolito.

\subsection{Análisis de Significancia Biológica: Cálculo de Fold
Change}\label{anuxe1lisis-de-significancia-bioluxf3gica-cuxe1lculo-de-fold-change}

\begin{Shaded}
\begin{Highlighting}[]
\CommentTok{\# Calcular fold change}
\NormalTok{results }\OtherTok{\textless{}{-}}\NormalTok{ results }\SpecialCharTok{\%\textgreater{}\%}
  \FunctionTok{mutate}\NormalTok{(}
    \AttributeTok{fold\_change\_GC\_HE =} \FunctionTok{apply}\NormalTok{(data\_matrix, }\DecValTok{1}\NormalTok{, }\ControlFlowTok{function}\NormalTok{(x) \{}
      \FunctionTok{mean}\NormalTok{(x[class\_labels }\SpecialCharTok{==} \StringTok{"GC"}\NormalTok{], }\AttributeTok{na.rm =} \ConstantTok{TRUE}\NormalTok{) }\SpecialCharTok{/} \FunctionTok{mean}\NormalTok{(x[class\_labels }\SpecialCharTok{==} \StringTok{"HE"}\NormalTok{], }\AttributeTok{na.rm =} \ConstantTok{TRUE}\NormalTok{)}
\NormalTok{    \}),}
    \AttributeTok{fold\_change\_GC\_BN =} \FunctionTok{apply}\NormalTok{(data\_matrix, }\DecValTok{1}\NormalTok{, }\ControlFlowTok{function}\NormalTok{(x) \{}
      \FunctionTok{mean}\NormalTok{(x[class\_labels }\SpecialCharTok{==} \StringTok{"GC"}\NormalTok{], }\AttributeTok{na.rm =} \ConstantTok{TRUE}\NormalTok{) }\SpecialCharTok{/} \FunctionTok{mean}\NormalTok{(x[class\_labels }\SpecialCharTok{==} \StringTok{"BN"}\NormalTok{], }\AttributeTok{na.rm =} \ConstantTok{TRUE}\NormalTok{)}
\NormalTok{    \})}
\NormalTok{  )}

\CommentTok{\# Mostrar los primeros resultados}
\FunctionTok{print}\NormalTok{(results)}
\end{Highlighting}
\end{Shaded}

\begin{verbatim}
##    Metabolite      p_GC_HE      p_GC_BN  p_adj_GC_HE p_adj_GC_BN
## 1          M4 4.229112e-03 0.0008962928 2.848735e-02  0.04660722
## 2          M5 3.026708e-01 0.7946699775 5.952753e-01  0.96708271
## 3          M7 4.901039e-03 0.0239424072 2.848735e-02  0.33621232
## 4          M8 4.900314e-02 0.0258624863 1.592602e-01  0.33621232
## 5         M11 7.325709e-01 0.7109812185 9.037709e-01  0.96096604
## 6         M14 1.380516e-01 0.0688759505 3.778253e-01  0.51164992
## 7         M15 6.717219e-01 0.2254644232 9.037709e-01  0.68965588
## 8         M25 7.629027e-02 0.6258443649 2.333585e-01  0.95717373
## 9         M26 2.331754e-01 0.9283069739 5.271791e-01  0.99638411
## 10        M31 9.781868e-01 0.7207245276 9.973670e-01  0.96096604
## 11        M32 4.930502e-03 0.1803395156 2.848735e-02  0.68912042
## 12        M33 5.464179e-01 0.9674657614 8.610221e-01  0.99638411
## 13        M36 2.614856e-01 0.6237016753 5.665522e-01  0.95717373
## 14        M37 9.636541e-01 0.5294535766 9.973670e-01  0.88811568
## 15        M45 1.903161e-03 0.4219280282 1.979287e-02  0.84385606
## 16        M48 1.768740e-01 0.1367737949 4.211816e-01  0.68912042
## 17        M50 6.917799e-01 0.7841530415 9.037709e-01  0.96708271
## 18        M51 6.952387e-01 0.1615040572 9.037709e-01  0.68912042
## 19        M65 6.985140e-01 0.1848310963 9.037709e-01  0.68912042
## 20        M66 5.235058e-01 0.9963841054 8.506969e-01  0.99638411
## 21        M68 8.565865e-01 0.7997030069 9.499488e-01  0.96708271
## 22        M71 4.799977e-01 0.1956116797 8.291136e-01  0.68912042
## 23        M73 9.178830e-02 0.9963841054 2.651662e-01  0.99638411
## 24        M74 7.647292e-01 0.5140603693 9.037709e-01  0.88811568
## 25        M75 4.942792e-01 0.3156787680 8.291136e-01  0.83482747
## 26        M88 1.000000e+00 0.2691903574 1.000000e+00  0.77766103
## 27        M89 1.837302e-05 0.4707569848 4.776986e-04  0.88092217
## 28        M90 8.589462e-01 0.1987847368 9.499488e-01  0.68912042
## 29        M91 4.295074e-02 0.5689498540 1.488959e-01  0.92454351
## 30        M93 9.491329e-01 0.2237300316 9.973670e-01  0.68965588
## 31       M101 1.585088e-01 0.4743427080 4.121230e-01  0.88092217
## 32       M104 3.204627e-01 0.5029545835 5.952753e-01  0.88811568
## 33       M105 6.025665e-01 0.9593073464 9.037709e-01  0.99638411
## 34       M106 7.359643e-01 0.3853049856 9.037709e-01  0.83482747
## 35       M107 6.535623e-01 0.8525873389 9.037709e-01  0.99638411
## 36       M110 2.271063e-02 0.9322469632 9.899537e-02  0.99638411
## 37       M115 1.781922e-01 0.7155922935 4.211816e-01  0.96096604
## 38       M116 8.346575e-01 0.4169485942 9.499488e-01  0.84385606
## 39       M118 9.996093e-04 0.9703994559 1.299492e-02  0.99638411
## 40       M119 3.568471e-01 0.7086582092 6.398637e-01  0.96096604
## 41       M120 9.892548e-03 0.6745610385 5.144125e-02  0.96096604
## 42       M122 7.566587e-01 0.3851279539 9.037709e-01  0.83482747
## 43       M126 3.946278e-02 0.0231141697 1.465761e-01  0.33621232
## 44       M129 6.733173e-01 0.8739662308 9.037709e-01  0.99638411
## 45       M130 2.913906e-02 0.0652119695 1.165562e-01  0.51164992
## 46       M134 1.406853e-04 0.0508401981 2.438545e-03  0.51164992
## 47       M137 3.205328e-01 0.3586951097 5.952753e-01  0.83482747
## 48       M138 8.335364e-09 0.0793265594 4.334389e-07  0.51562264
## 49       M142 2.901147e-03 0.3277100198 2.514328e-02  0.83482747
## 50       M144 3.051509e-01 0.7497147605 5.952753e-01  0.96708271
## 51       M148 2.284509e-02 0.3386046293 9.899537e-02  0.83482747
## 52       M149 8.768758e-01 0.1018612947 9.499488e-01  0.58853192
##    fold_change_GC_HE fold_change_GC_BN
## 1          0.5117520         0.4626141
## 2          1.5603014         0.9316429
## 3          2.1954394         2.0281397
## 4          0.6862251         0.6827797
## 5          1.1755284         0.7790021
## 6          0.7333716         0.7944306
## 7          0.9385569         0.9364900
## 8          1.4676852         1.3224063
## 9          1.7509244         1.4935404
## 10         1.3892538         0.8307738
## 11         1.9033130         1.2510197
## 12         1.1169765         1.2001980
## 13         1.2927439         1.1048997
## 14         1.3166263         0.9428547
## 15         0.5154190         0.8511128
## 16         0.7524176         0.8503668
## 17         1.6146442         0.5215226
## 18         0.8421755         0.5917582
## 19         1.2483245         0.8745886
## 20         1.6622832         1.7019147
## 21         0.8148371         0.7166607
## 22         0.8383298         1.0353402
## 23         1.3565423         1.0098982
## 24         1.0791848         1.2060560
## 25         1.2179352         0.8229847
## 26         1.0266053         0.8681927
## 27         2.5536808         1.0674063
## 28         0.9523501         0.6982808
## 29         1.4125942         1.2643069
## 30         1.0393082         0.7912665
## 31         0.8108182         0.9149028
## 32         1.2936323         0.9819485
## 33         0.6401595         0.9617440
## 34         1.0024657         0.7897173
## 35         1.0714657         0.9848978
## 36         1.2707657         1.0695005
## 37         1.4469282         1.8571535
## 38         1.3164273         0.8383684
## 39         2.8159696         1.1064928
## 40         1.0826111         0.9233699
## 41         1.2585782         1.0923453
## 42         1.0426013         0.8883268
## 43         2.1154924         2.0314376
## 44         1.1166486         1.2167529
## 45         1.3247370         1.9438442
## 46         2.9017333         1.5939077
## 47         0.8678338         0.8179556
## 48         4.5685174         1.3034061
## 49         2.3808500         1.0739792
## 50         1.4125358         0.6196860
## 51         1.4607080         1.3019356
## 52         0.9337956         0.8200261
\end{verbatim}

Interpretación: Un ``fold change'' mayor que 1 indica que el metabolito
es más abundante en el grupo GC en comparación con HE o BN, mientras que
un valor menor que 1 indica menor abundancia en GC.

\subsection{Analisis Multivariado}\label{analisis-multivariado}

\begin{enumerate}
\def\labelenumi{\arabic{enumi}.}
\tightlist
\item
  PLS-DA (Partial Least Squares Discriminant Analysis)
\end{enumerate}

\begin{Shaded}
\begin{Highlighting}[]
\CommentTok{\# Instala mixOmics si no lo tienes}
\ControlFlowTok{if}\NormalTok{ (}\SpecialCharTok{!}\FunctionTok{requireNamespace}\NormalTok{(}\StringTok{"BiocManager"}\NormalTok{, }\AttributeTok{quietly =} \ConstantTok{TRUE}\NormalTok{))}
    \FunctionTok{install.packages}\NormalTok{(}\StringTok{"BiocManager"}\NormalTok{)}
\NormalTok{BiocManager}\SpecialCharTok{::}\FunctionTok{install}\NormalTok{(}\StringTok{"mixOmics"}\NormalTok{)}
\end{Highlighting}
\end{Shaded}

\begin{verbatim}
## Bioconductor version 3.19 (BiocManager 1.30.25), R 4.4.1 (2024-06-14 ucrt)
\end{verbatim}

\begin{verbatim}
## Installation paths not writeable, unable to update packages
##   path: C:/Program Files/R/R-4.4.1/library
##   packages:
##     boot, foreign, MASS, Matrix, nlme, survival
\end{verbatim}

\begin{verbatim}
## Old packages: 'curl', 'xfun'
\end{verbatim}

\begin{Shaded}
\begin{Highlighting}[]
\FunctionTok{library}\NormalTok{(mixOmics)}
\end{Highlighting}
\end{Shaded}

\begin{verbatim}
## Cargando paquete requerido: MASS
\end{verbatim}

\begin{verbatim}
## 
## Adjuntando el paquete: 'MASS'
\end{verbatim}

\begin{verbatim}
## The following object is masked from 'package:dplyr':
## 
##     select
\end{verbatim}

\begin{verbatim}
## Cargando paquete requerido: lattice
\end{verbatim}

\begin{verbatim}
## 
## Loaded mixOmics 6.28.0
## Thank you for using mixOmics!
## Tutorials: http://mixomics.org
## Bookdown vignette: https://mixomicsteam.github.io/Bookdown
## Questions, issues: Follow the prompts at http://mixomics.org/contact-us
## Cite us:  citation('mixOmics')
\end{verbatim}

Configuramos los datos para PLS-DA

\begin{Shaded}
\begin{Highlighting}[]
\CommentTok{\# Extraer la matriz de datos y etiquetas de clase}
\NormalTok{data\_matrix }\OtherTok{\textless{}{-}} \FunctionTok{t}\NormalTok{(}\FunctionTok{assay}\NormalTok{(se\_filtered, }\StringTok{"counts"}\NormalTok{))  }\CommentTok{\# Transponer para que las muestras sean filas}
\NormalTok{class\_labels }\OtherTok{\textless{}{-}} \FunctionTok{colData}\NormalTok{(se\_filtered)}\SpecialCharTok{$}\NormalTok{Class}

\CommentTok{\# Convertir las etiquetas de clase a un factor}
\NormalTok{class\_labels }\OtherTok{\textless{}{-}} \FunctionTok{as.factor}\NormalTok{(class\_labels)}
\end{Highlighting}
\end{Shaded}

Ralizamos el PLS

\begin{Shaded}
\begin{Highlighting}[]
\CommentTok{\# Ejecutar PLS{-}DA}
\NormalTok{plsda\_result }\OtherTok{\textless{}{-}} \FunctionTok{plsda}\NormalTok{(}\AttributeTok{X =}\NormalTok{ data\_matrix, }\AttributeTok{Y =}\NormalTok{ class\_labels, }\AttributeTok{ncomp =} \DecValTok{2}\NormalTok{)  }\CommentTok{\# ncomp = 2 componentes}

\CommentTok{\# Graficar el PLS{-}DA}
\FunctionTok{plotIndiv}\NormalTok{(plsda\_result, }\AttributeTok{group =}\NormalTok{ class\_labels, }\AttributeTok{legend =} \ConstantTok{TRUE}\NormalTok{, }
          \AttributeTok{title =} \StringTok{"PLS{-}DA de Metabolitos"}\NormalTok{, }\AttributeTok{ellipse =} \ConstantTok{TRUE}\NormalTok{, }\AttributeTok{comp =} \FunctionTok{c}\NormalTok{(}\DecValTok{1}\NormalTok{, }\DecValTok{2}\NormalTok{))}
\end{Highlighting}
\end{Shaded}

\includegraphics{PEC1_Datos_Omicos_files/figure-latex/unnamed-chunk-25-1.pdf}
Interpretación: plotIndiv muestra la discriminación entre clases. Si el
PLS-DA separa bien los grupos, debería haber una clara diferencia en el
gráfico.

Evaluar la Robustez del Modelo con Validación Cruzad

\begin{Shaded}
\begin{Highlighting}[]
\CommentTok{\# Validación cruzada con mixOmics}
\FunctionTok{set.seed}\NormalTok{(}\DecValTok{625745221}\NormalTok{)  }\CommentTok{\# Para reproducibilidad}
\NormalTok{cv\_result }\OtherTok{\textless{}{-}} \FunctionTok{perf}\NormalTok{(plsda\_result, }\AttributeTok{validation =} \StringTok{"Mfold"}\NormalTok{, }\AttributeTok{folds =} \DecValTok{5}\NormalTok{, }\AttributeTok{progressBar =} \ConstantTok{TRUE}\NormalTok{, }\AttributeTok{nrepeat =} \DecValTok{10}\NormalTok{)}
\end{Highlighting}
\end{Shaded}

\begin{verbatim}
## 
## comp 1 
##   |                                                                              |                                                                      |   0%  |                                                                              |=                                                                     |   2%  |                                                                              |===                                                                   |   4%  |                                                                              |====                                                                  |   6%  |                                                                              |======                                                                |   8%  |                                                                              |=======                                                               |  10%  |                                                                              |========                                                              |  12%  |                                                                              |==========                                                            |  14%  |                                                                              |===========                                                           |  16%  |                                                                              |=============                                                         |  18%  |                                                                              |==============                                                        |  20%  |                                                                              |===============                                                       |  22%  |                                                                              |=================                                                     |  24%  |                                                                              |==================                                                    |  26%  |                                                                              |====================                                                  |  28%  |                                                                              |=====================                                                 |  30%  |                                                                              |======================                                                |  32%  |                                                                              |========================                                              |  34%  |                                                                              |=========================                                             |  36%  |                                                                              |===========================                                           |  38%  |                                                                              |============================                                          |  40%  |                                                                              |=============================                                         |  42%  |                                                                              |===============================                                       |  44%  |                                                                              |================================                                      |  46%  |                                                                              |==================================                                    |  48%  |                                                                              |===================================                                   |  50%  |                                                                              |====================================                                  |  52%  |                                                                              |======================================                                |  54%  |                                                                              |=======================================                               |  56%  |                                                                              |=========================================                             |  58%  |                                                                              |==========================================                            |  60%  |                                                                              |===========================================                           |  62%  |                                                                              |=============================================                         |  64%  |                                                                              |==============================================                        |  66%  |                                                                              |================================================                      |  68%  |                                                                              |=================================================                     |  70%  |                                                                              |==================================================                    |  72%  |                                                                              |====================================================                  |  74%  |                                                                              |=====================================================                 |  76%  |                                                                              |=======================================================               |  78%  |                                                                              |========================================================              |  80%  |                                                                              |=========================================================             |  82%  |                                                                              |===========================================================           |  84%  |                                                                              |============================================================          |  86%  |                                                                              |==============================================================        |  88%  |                                                                              |===============================================================       |  90%  |                                                                              |================================================================      |  92%  |                                                                              |==================================================================    |  94%  |                                                                              |===================================================================   |  96%  |                                                                              |===================================================================== |  98%  |                                                                              |======================================================================| 100%
## comp 2 
##   |                                                                              |                                                                      |   0%  |                                                                              |=                                                                     |   2%  |                                                                              |===                                                                   |   4%  |                                                                              |====                                                                  |   6%  |                                                                              |======                                                                |   8%  |                                                                              |=======                                                               |  10%  |                                                                              |========                                                              |  12%  |                                                                              |==========                                                            |  14%  |                                                                              |===========                                                           |  16%  |                                                                              |=============                                                         |  18%  |                                                                              |==============                                                        |  20%  |                                                                              |===============                                                       |  22%  |                                                                              |=================                                                     |  24%  |                                                                              |==================                                                    |  26%  |                                                                              |====================                                                  |  28%  |                                                                              |=====================                                                 |  30%  |                                                                              |======================                                                |  32%  |                                                                              |========================                                              |  34%  |                                                                              |=========================                                             |  36%  |                                                                              |===========================                                           |  38%  |                                                                              |============================                                          |  40%  |                                                                              |=============================                                         |  42%  |                                                                              |===============================                                       |  44%  |                                                                              |================================                                      |  46%  |                                                                              |==================================                                    |  48%  |                                                                              |===================================                                   |  50%  |                                                                              |====================================                                  |  52%  |                                                                              |======================================                                |  54%  |                                                                              |=======================================                               |  56%  |                                                                              |=========================================                             |  58%  |                                                                              |==========================================                            |  60%  |                                                                              |===========================================                           |  62%  |                                                                              |=============================================                         |  64%  |                                                                              |==============================================                        |  66%  |                                                                              |================================================                      |  68%  |                                                                              |=================================================                     |  70%  |                                                                              |==================================================                    |  72%  |                                                                              |====================================================                  |  74%  |                                                                              |=====================================================                 |  76%  |                                                                              |=======================================================               |  78%  |                                                                              |========================================================              |  80%  |                                                                              |=========================================================             |  82%  |                                                                              |===========================================================           |  84%  |                                                                              |============================================================          |  86%  |                                                                              |==============================================================        |  88%  |                                                                              |===============================================================       |  90%  |                                                                              |================================================================      |  92%  |                                                                              |==================================================================    |  94%  |                                                                              |===================================================================   |  96%  |                                                                              |===================================================================== |  98%  |                                                                              |======================================================================| 100%
\end{verbatim}

\begin{Shaded}
\begin{Highlighting}[]
\CommentTok{\# Ver resultados de validación cruzada}
\FunctionTok{print}\NormalTok{(cv\_result)}
\end{Highlighting}
\end{Shaded}

\begin{verbatim}
## 
## Call:
##  perf.mixo_plsda(object = plsda_result, validation = "Mfold", folds = 5, nrepeat = 10, progressBar = TRUE) 
## 
##  Main numerical outputs: 
##  -------------------- 
##  Error rate (overall or BER) for each component and for each distance: see object$error.rate 
##  Error rate per class, for each component and for each distance: see object$error.rate.class 
##  Prediction values for each component: see object$predict 
##  Classification of each sample, for each component and for each distance: see object$class 
##  AUC values: see object$auc if auc = TRUE 
## 
##  Visualisation Functions: 
##  -------------------- 
##  plot
\end{verbatim}

error rate

\begin{Shaded}
\begin{Highlighting}[]
\CommentTok{\# Ver tasa de error}
\FunctionTok{print}\NormalTok{(cv\_result}\SpecialCharTok{$}\NormalTok{error.rate)}
\end{Highlighting}
\end{Shaded}

\begin{verbatim}
## $overall
##        max.dist centroids.dist mahalanobis.dist
## comp1 0.5243902      0.5382114        0.5382114
## comp2 0.4130081      0.4601626        0.4219512
## 
## $BER
##        max.dist centroids.dist mahalanobis.dist
## comp1 0.5276163      0.5375388        0.5375388
## comp2 0.4192636      0.4642829        0.4251744
\end{verbatim}

Error rate por clase

\begin{Shaded}
\begin{Highlighting}[]
\CommentTok{\# Tasa de error por clase}
\FunctionTok{print}\NormalTok{(cv\_result}\SpecialCharTok{$}\NormalTok{error.rate.class)}
\end{Highlighting}
\end{Shaded}

\begin{verbatim}
## $max.dist
##        comp1     comp2
## BN 1.0000000 0.9300000
## GC 0.3953488 0.1627907
## HE 0.1875000 0.1650000
## 
## $centroids.dist
##        comp1     comp2
## BN 0.7375000 0.6200000
## GC 0.5651163 0.2953488
## HE 0.3100000 0.4775000
## 
## $mahalanobis.dist
##        comp1     comp2
## BN 0.7375000 0.6725000
## GC 0.5651163 0.2930233
## HE 0.3100000 0.3100000
\end{verbatim}

Validación Cruzada: El resultado de la validación cruzada (cv\_result)
ayuda a verificar si el modelo es robusto y no sobreajustado.

\begin{Shaded}
\begin{Highlighting}[]
\CommentTok{\# Graficar resultados de validación cruzada}
\FunctionTok{plot}\NormalTok{(cv\_result)}
\end{Highlighting}
\end{Shaded}

\includegraphics{PEC1_Datos_Omicos_files/figure-latex/unnamed-chunk-29-1.pdf}
Esto generará gráficos que te permiten ver cómo varía el error con el
número de componentes, lo cual es útil para decidir cuántos componentes
incluir en el modelo PLS-DA final.

Analizar las Cargas (Loadings) para Identificar Metabolitos Relevantes

\begin{Shaded}
\begin{Highlighting}[]
\CommentTok{\# Extraer y visualizar las cargas para interpretar los metabolitos que más contribuyen}
\NormalTok{loadings }\OtherTok{\textless{}{-}}\NormalTok{ plsda\_result}\SpecialCharTok{$}\NormalTok{loadings}\SpecialCharTok{$}\NormalTok{X  }\CommentTok{\# Cargas para las variables}

\CommentTok{\# Mostrar los primeros metabolitos relevantes}
\NormalTok{top\_metabolites }\OtherTok{\textless{}{-}} \FunctionTok{rownames}\NormalTok{(loadings)[}\FunctionTok{order}\NormalTok{(}\FunctionTok{abs}\NormalTok{(loadings[, }\DecValTok{1}\NormalTok{]), }\AttributeTok{decreasing =} \ConstantTok{TRUE}\NormalTok{)[}\DecValTok{1}\SpecialCharTok{:}\DecValTok{10}\NormalTok{]]}
\FunctionTok{print}\NormalTok{(top\_metabolites)}
\end{Highlighting}
\end{Shaded}

\begin{verbatim}
##  [1] "M138" "M134" "M118" "M45"  "M89"  "M32"  "M7"   "M126" "M4"   "M91"
\end{verbatim}

\begin{Shaded}
\begin{Highlighting}[]
\CommentTok{\# Crear un data frame con IDs y Labels de los metabolitos en rowData}
\NormalTok{metabolite\_info }\OtherTok{\textless{}{-}} \FunctionTok{data.frame}\NormalTok{(}
  \AttributeTok{ID =} \FunctionTok{rownames}\NormalTok{(}\FunctionTok{rowData}\NormalTok{(se\_filtered)),}
  \AttributeTok{Label =} \FunctionTok{rowData}\NormalTok{(se\_filtered)}\SpecialCharTok{$}\NormalTok{Label}
\NormalTok{)}

\CommentTok{\# Filtrar para obtener solo los metabolitos en top\_metabolites y asegurar el orden correcto}
\NormalTok{top\_metabolite\_info }\OtherTok{\textless{}{-}}\NormalTok{ metabolite\_info[}\FunctionTok{match}\NormalTok{(top\_metabolites, metabolite\_info}\SpecialCharTok{$}\NormalTok{ID), ]}

\CommentTok{\# Mostrar los resultados correctamente alineados}
\FunctionTok{print}\NormalTok{(top\_metabolite\_info)}
\end{Highlighting}
\end{Shaded}

\begin{verbatim}
##      ID                       Label
## 48 M138                        u233
## 46 M134                        u144
## 39 M118                     Tropate
## 15  M45                     Citrate
## 27  M89 N-AcetylglutamineDerivative
## 11  M32                     Alanine
## 3    M7             2-Furoylglycine
## 43 M126             trans-Aconitate
## 1    M4        1-Methylnicotinamide
## 29  M91           N-Acetylserotonin
\end{verbatim}

\section{\texorpdfstring{\emph{Discusión, Limitaciones y conclusiones
del
estudio}}{Discusión, Limitaciones y conclusiones del estudio}}\label{discusiuxf3n-limitaciones-y-conclusiones-del-estudio}

Según análisis de la varianza, se reveló diferencias significativas en
la variabilidad de metabolitos específicos entre grupos, evaluados a
través del test de Bartlett. Por ejemplo, los metabolitos M4, M5, M7, y
M8 presentaron p-valores muy bajos (e.g., M4 con 𝑝\textless4.2×10−7
p\textless4.2×10−7), sugiriendo una variabilidad significativa en cáncer
gástrico.

Como se pudo observar en el análisis de Significancia Biológica y Fold
Change, hubo un aumento en la concentración de varios metabolitos en
cáncer gástrico en comparación con controles sanos. El fold change de
M4, por ejemplo, mostró un aumento significativo en GC respecto a HE,
con valores ajustados de p = 0.028 para la comparación GC-HE.

La proyección de las muestras en un análisis de componentes principales
(PCA) indicó una clara separación entre los grupos. Además, el modelo
PLS-DA evidenció una separación robusta entre GC, BN y HE. Los
metabolitos M138, M134 y M118 se destacaron como importantes en la
discriminación entre grupos.

Validación del Modelo La validación cruzada del modelo PLS-DA mostró una
tasa de error de clasificación moderada. Para el componente principal 1,
el error de clasificación promedio fue de 0.52 para GC frente a HE y de
0.41 en el segundo componente. Este resultado sugiere que el modelo
presenta una capacidad para discriminar razonable, para distinguir entre
cáncer gástrico y otros estados clínicos, aunque podría beneficiarse de
otras optimizaciones adicionales.

Así pues para finalizar, este estudio ha identificado metabolitos
diferenciadores en cáncer gástrico, proporcionando un marco preliminar
para el desarrollo de biomarcadores diagnósticos. Los hallazgos indican
una alteración en el perfil metabólico de los pacientes con GC,
especialmente en metabolitos como M4(1-Methylnicotinamide ), M7
(2-Furoylglycine) y M138 (metabolito u233), que podrían ser interesantes
para seguir trabajando en ellos en estudios adicionales.

Sin embargo, sería crucial validar estos resultados en cohortes más
amplias y en otros contextos clínicos para confirmar su aplicabilidad.

\section{\texorpdfstring{\emph{Apendices}}{Apendices}}\label{apendices}

Este estudio se encuentra en el repositorio github
\url{https://github.com/GilCaraballo/PEC1_Datos_Omicos}

\end{document}
